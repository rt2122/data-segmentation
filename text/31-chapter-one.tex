\chapter{Постановка задачи}
\label{cha:ch_1}

В более формальном и подробном виде задачу можно описать так: при имеющихся обработанных данных, 
полученных из космических обзоров, требуется получить матрицы сегментации, где для каждого пикселя 
матрицы мы будем иметь информацию о вероятности, с которой в данном пикселе находится скопление. 
Впоследствии координаты пикселей изображений можно преобразовать в небесные координаты способом, 
обратным к способу, с помощью которого были получены изображения космических обзоров. На полученных 
масках можно будет детектировать скопления.Главной целью является создание итогового каталога 
скоплений, найденных в разных диапазонах.\\

Пока что не существует какого-то универсального метода для сегментации и детекции скоплений 
на оптических данных, и есть возможность применить методы глубокого обучения в данной области.\\

