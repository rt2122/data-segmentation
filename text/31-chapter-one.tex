\chapter{Постановка задачи}
\label{cha:ch_1}
На данный момент имеется доступ к оптическим данным, полученным с помошью телескопа PS1. Выбрана 
архитектура нейросетевой модели для сегментации изображений. Требуется обработать данные PS1, 
преобразовать их в подходящий для нейросети вид, и получить алгоритм сегментации скоплений 
галактик.\\

В более формальном и подробном виде задачу можно описать так: при имеющихся обработанных данных, 
полученных из космических обзоров, требуется получить матрицы сегментации, где для каждого пикселя 
матрицы мы будем иметь информацию о вероятности, с которой в данном пикселе находится скопление. 
Впоследствии координаты пикселей изображений можно преобразовать в небесные координаты способом, 
обратным к способу, с помощью которого были получены изображения космических обзоров.\\

Для достижения цели задание было разделено на несколько шагов:

\begin{enumerate}
    \item Создание простейших симуляций рентгеновских данных.
    \item Создание образца модели U-net.
    \item Проверка работы U-net на данных симуляций.
    \item Загрузка и обработка данных о скоплениях.
    \item Генерация <<патчей>> --- небольших областей неба, на которых будет тренироваться нейросеть.
    \item Загрузка и обработка обзоров неба PanSTARRS1 из области патчей.
    \item Преобразование данных PanSTARRS1 в двумерные матрицы для загрузки в нейросеть.
    \item Обучение модели, подбор параметров модели (количество слоёв, методы аугментации, размер 
        батча, количество эпох обучения)
    \item Тестирование модели на заранее выбранных данных.
    \item Преобразование масок сегментации в координаты (детектирование скоплений).
    \item Сравнение полученных скоплений с существующими каталогами.
\end{enumerate}

Пока что не существует какого-то универсального метода для сегментации и детекции скоплений 
на оптических данных, и есть возможность применить методы глубокого обучения в данной области.\\

Кроме того, сами по себе данные телескопов не похожи на обычные изображения, которые хранятся в 
матрицах из одного или трех фильтров и используют однобайтовые целые значения, не требующие 
большого количества памяти. Матрицы, получающиеся при преобразовании данных из космических 
координат, обычно получаются разреженными и для них приходится использовать значения с плавающей 
точкой. Более того, нужно учитывать точность преобразования и выбирать достаточно детализированное 
разбиение проекции на пиксели изображения, иначе разные объекты могут слиться в один. Угловой 
размер области тоже имеет значение, так как на слишком маленьких частях неба искать такие большие 
объекты, как скопления галактик, бессмысленно.\\
