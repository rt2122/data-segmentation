\chapter{Постановка задачи}
\label{cha:ch_1}
%Формальная постановка задачи сегментации
%Более четко сформулировать
На данный момент имеется доступ к оптическим данным, полученным с помошью телескопа PS1. Выбрана 
архитектура нейросетевой модели для сегментации изображений. Требуется обработать данные PS1, 
преобразовать их в подходящий для нейросети вид, и получить алгоритм сегментации скоплений галактик.
Для достижения цели задание было разделено на несколько шагов:

\begin{enumerate}
    \item Создание простейших симуляций рентгеновских данных.
    \item Создание образца модели U-net.
    \item Проверка работы U-net на данных симуляций.
    \item Загрузка и обработка данных о скоплениях.
    \item Генерация <<патчей>> --- небольших областей неба, на которых будет тренироваться нейросеть.
    \item Загрузка и обработка обзоров неба PanSTARRS1 из области патчей.
    \item Преобразование данных PanSTARRS1 в двумерные матрицы для загрузки в нейросеть.
    \item Обучение модели, подбор параметров модели (количество слоёв, методы аугментации, размер 
        батча, количество эпох обучения)
    \item Тестирование модели на заранее выбранных данных.
    \item Преобразование масок сегментации в координаты (детектирование скоплений).
    \item Сравнение полученных скоплений с существующими каталогами.
\end{enumerate}

Пока что не существует какого-то универсального метода для сегментации и детекции скоплений 
на оптических данных, и есть возможность применить методы глубокого обучения в данной области.\\
