\chapter{Постановка задачи}
\label{cha:ch_1}
Эта работа во многом является повторением исследования из статьи о детекции эффекта 
Сюняева-Зельдовича \cite{Bonjean}, с той разницей, что здесь будут использоваться оптические данные, в то время 
как в упомянутой статье использовались данные микроволнового диапазона.\\

Основной задачей этой работы является создание нейросетевой модели, способной сегментировать  
данные космических обзоров так, чтобы на них можно было детектировать объекты определённого типа 
(в данном случае скопления). В лучшем случае модель должна будет демонстрировать результаты, 
превосходящие по качеству методы, использующиеся для решения аналогичных проблем.
