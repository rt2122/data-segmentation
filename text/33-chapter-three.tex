\chapter{Исследование и построение решения задачи}
\label{cha:ch_3}
Так же, как и в статье \cite{Bonjean}, в качестве списка скоплений будем использовать каталог PSZ2. 
Аналогичным образом будем генерировать патчи для создания тренировочных, валидационных и тестовых 
выборок (в том числе для валидации и тестирования будут выбраны те же пиксели разбиения $n_{side}=2$. 
Патчи выбирались так, чтобы в их окрестности находился хотя бы одно из скоплений нужного каталога. 
После этого в базе данных PS1 (Pan-STARRS1) запрашивался список объектов, подходящих под заданные условия. \\

После того, как будут получены данные для обучения, их нужно из таблиц преобразовать в двухмерные 
матрицы изображений, чтобы создать выборки с количеством каналов, совпадающих с количеством 
исследуемых параметров у объектов. Перед этим нужно удалить из таблицы повторяющиеся объекты. \\

Для примера, таблица с данными для 100 патчей содержит около 10 миллионов объектов. Обработка 
сотни объектов будет длится несколько минут, поэтому необходимо отдельно распараллелить процесс.\\
