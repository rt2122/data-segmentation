\Introduction
В 2019 году произошел запуск космической обсерватории СРГ (Спектр-Рентген-Гамма) с телескопами 
eROSITA и ART-XC на борту. Основной задачей этих телескопов является создание обзора всего неба в 
рентгеновском диапазоне. Данные, полученные от этих телескопов будут использоваться для обнаружения 
астрономических объектов трёх категорий:\\
\begin{enumerate}
    \item Скопления галактик.\\
    \item Сверхмассивные чёрные дыры.\\
    \item Рентгеновские звёзды в галактике Млечный путь. \\
\end{enumerate}
Полные обзоры неба, полученные телескопом eROSITA, появятся к июню 2020 года, поэтому на данный 
момент есть возможность подготовить модели для сегментации данных на примере других диапазонов.\\
\\
В первую очередь будут использоваться данные оптического диапазона. Видимое излучение --- тот 
диапазон частот, что доступен глазу человека. На текущий момент существует большое количество 
оптических телескопов, и, как следствие, большое количество данных, извлеченных с их помощью. В 
данной работе будут использоваться данные телескопа Pan-STARRS1, который является частью системы 
телескопов Pan-STARRS (Panoramic Survey Telescope and Rapid Response System). Этот телескоп 
построен на вершине гавайского вулкана Халеакала. На 2007 год он обладал самой большой 
светочувствительной матрицей в мире. Кроме того, его данные находятся в общем доступе.\\
\\
