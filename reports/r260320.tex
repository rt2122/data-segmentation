\documentclass{article}
  \usepackage[utf8]{inputenc}
  \usepackage[russian]{babel}
  \usepackage[left=2cm,right=2cm,
    top=2cm,bottom=2cm,bindingoffset=0cm]{geometry}

\begin{document}
Отчет по курсовой работе за неделю (исправленный)\\
Дата: 26.3.2020\\
Научные руководители: Герасимов С.В., Мещеряков А.В.\\
Студент: Немешаева Алиса\\
Курс: 3\\

За прошедшую неделю была проведена следующая работа:\\
\begin{enumerate}
    \item Изменен вектор работы из-за невозможности получить доступ к данным из ИКИ РАН \\
    \item Изучены данные из статьи об использовании глубокого обучения для обнаружения 
        эффекта Сюняева — Зельдовича: \\
        \begin{enumerate}
            \item каталог кластеров PSZ2 \\
                   https://heasarc.gsfc.nasa.gov/W3Browse/all/plancksz2.html\\
            \item MCXC (Мета-каталог обнаруженных кластеров в рентгеновском диапазоне) \\
                   https://heasarc.gsfc.nasa.gov/W3Browse/rosat/mcxc.html\\
            \item каталог кластеров RedMaPPer \\
                   http://risa.stanford.edu/redmapper/v6.3/redmapper\_dr8\_public\_v6.3\_catalog.fits.gz \\
        \end{enumerate}
    \item Эти данные будут использоваться в качестве целевых признаков для обучения. \\
    \item Упомянутые данные в совокупности содержат большое количество источников в рентгеновском 
    диапазоне, следовательно, при решении задачи сегментации и детекции рентгеновских источников они 
    подходят для обучения лучше всего. \\
    \item Проведены сравнения по данным и их наполнениям \\
               https://github.com/rt2122/data-segmentation/blob/master/data\_src/table.ipynb \\
               https://github.com/rt2122/data-segmentation/blob/master/data\_src/diagram.ipynb \\ 
\end{enumerate}

Общее количество строк кода: 226\\
\end{document}
