\documentclass{article}
  \usepackage{amssymb}
  \usepackage{hyperref}
  \usepackage[utf8]{inputenc}
  \usepackage[russian]{babel}
  \usepackage[left=2cm,right=2cm,
    top=2cm,bottom=2cm,bindingoffset=0cm]{geometry}

\begin{document}
Отчет по курсовой работе за неделю\\
Дата: 16.4.2020\\
Научные руководители: Герасимов С.В., Мещеряков А.В.\\
Студент: Немешаева Алиса\\
Курс: 3\\

\renewcommand{\labelitemi}{$\blacksquare$}
\renewcommand\labelitemii{$\square$}
За прошедшую неделю была проведена следующая работа:\\
\begin{enumerate}
    \item Начата работа над текстом курсовой работы (описана общая структура и постановка задачи).\\
    \item \href{https://github.com/rt2122/data-segmentation/blob/master/old/healpy/make_pic_proper.ipynb}
        {Код} для определения координат пикселей патчей.\\
    \item Начата \href{https://github.com/rt2122/data-segmentation/blob/master/rk/calc_npix.ipynb}
        {обработка данных PS1} (оптимизация вычисления координат пикселей для объектов PS1).\\
\end{enumerate}

Общее количество строк кода: 221\\
\end{document}
