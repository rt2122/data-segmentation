\documentclass{article}
  \usepackage[utf8]{inputenc}
  \usepackage[russian]{babel}
  \usepackage[left=2cm,right=2cm,
    top=2cm,bottom=2cm,bindingoffset=0cm]{geometry}

\begin{document}
Отчет по курсовой работе за неделю\\
Дата: 02.4.2020\\
Научные руководители: Герасимов С.В., Мещеряков А.В.\\
Студент: Немешаева Алиса\\
Курс: 3\\

За прошедшую неделю была проведена следующая работа:\\
\begin{enumerate}
    \item Обработаны и сохранены данные Planck: \\
        \begin{itemize}
            \item https://github.com/rt2122/data-segmentation/blob/master/data\_src/planck\_z.csv\\ 
            \item https://github.com/rt2122/data-segmentation/blob/master/data\_src/planck\_no\_z.csv\\
            \item Данные разделены по значению красного смещения: у planck\_z красное смещение измерено 
                (то есть значение параметра z положительное), у объектов из planck\_no\_z этот параметр не
                измерен (значение z равно -1).\\
        \end{itemize}
    \item Обработаны и сохранены данные MCXC: \\
        \begin{itemize}
            \item https://github.com/rt2122/data-segmentation/blob/master/data\_src/mcxcwp.csv \\
            \item Из данных каталога MCXC удалены объекты, которые уже содержатся в каталоге 
                Planck (по названиям объектов).\\
        \end{itemize}
    \item Обработаны и сохранены данные RedMAPPer: \\
        \begin{itemize}
            \item https://github.com/rt2122/data-segmentation/blob/master/data\_src/rm30\_15130.csv \\
            \item https://github.com/rt2122/data-segmentation/blob/master/data\_src/rm50\_4711.csv \\
            \item Удалены объекты, содержащиеся в каталоге Planck (по названиям). \\
            \item Удалены объекты, содержащиеся в каталоге MCXC (в каталоге MCXC нет имен объектов 
                относительно RedMAPPer и наоборот, поэтому объекты исключались по расстоянию между ними). \\
            \item По параметру $\lambda$ каталог преобразован в два отдельных каталога: \\
                \begin{itemize}
                    \item $\lambda > 30 $\\
                    \item $\lambda > 50 $\\
                \end{itemize}
        \end{itemize}

\end{enumerate}

Общее количество строк кода: \\
\end{document}
