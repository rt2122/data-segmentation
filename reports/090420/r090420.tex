\documentclass{article}
  \usepackage{amssymb}
  \usepackage[utf8]{inputenc}
  \usepackage[russian]{babel}
  \usepackage[left=2cm,right=2cm,
    top=2cm,bottom=2cm,bindingoffset=0cm]{geometry}
  \usepackage{url} 

\begin{document}
Отчет по курсовой работе за неделю\\
Дата: 09.4.2020\\
Научные руководители: Герасимов С.В., Мещеряков А.В.\\
Студент: Немешаева Алиса\\
Курс: 3\\

\renewcommand{\labelitemi}{$\blacksquare$}
\renewcommand\labelitemii{$\square$}
За прошедшую неделю была проведена следующая работа:\\
\begin{enumerate}
    \item Базовый код для запроса данных из PanSTARRS1: \\
        \begin{itemize}
            \item \url{https://github.com/rt2122/data-segmentation/blob/master/healpy/clusters_in_pix.ipynb}\\
            \item Обработка списка источников с записью номера healpix-пикселя для каждого кластера. \\
            \item \url{https://github.com/rt2122/data-segmentation/blob/master/patches/generate_patches.ipynb} \\
            \item Создание "патчей" для формирования запроса. \\
            \item Патч - область со случайно выбранным центром и следующими свойствами: \\
                \begin{itemize}
                    \item Радиус патча - один градус.\\
                    \item В области патча находится хотя бы один из источников выбранного каталога (в данном случае planck\_z).\\
                \end{itemize}
        \end{itemize}
    \item Загружены данные из PanSTARRS1 для тестового healpix-пикселя.\\

\end{enumerate}

Общее количество строк кода: 202 \\
\end{document}
