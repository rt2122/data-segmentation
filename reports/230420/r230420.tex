\documentclass{article}
  \usepackage{amssymb}
  \usepackage{hyperref}
  \usepackage[utf8]{inputenc}
  \usepackage[russian]{babel}
  \usepackage[left=2cm,right=2cm,
    top=2cm,bottom=2cm,bindingoffset=0cm]{geometry}
  \usepackage{gensymb}
\begin{document}
Отчет по курсовой работе за неделю\\
Дата: 23.4.2020\\
Научные руководители: Герасимов С.В., Мещеряков А.В.\\
Студент: Немешаева Алиса\\
Курс: 3\\

\renewcommand{\labelitemi}{$\blacksquare$}
\renewcommand\labelitemii{$\square$}
За прошедшую неделю была проведена следующая работа:\\
\begin{enumerate}
    \item \href{https://github.com/rt2122/data-segmentation/blob/master/rk/
        find_clusters_in_patch.ipynb}{Функция} по созданию матриц сегментации (последний этап 
        создания тренирововчных данных). \\
    \item Все функции по обработке данных собраны в одном \href{https://github.com/rt2122/
        data-segmentation/blob/master/rk/all_p.py}{файле} и перепроверены.\\
    \item Исправлен размер патча: $1.7 \degree$ (старый был $1.45 \degree$ из-за ошибки). Новое 
        разрешение матриц с данными: 4096 x 4096 в разбиении $n_{side} = 2 ^ {17}$ (старые параметры 
        64 x 64 и $2 ^ {11}$ соответственно). Размер части неба, отражаемой патчем не изменился, но
        теперь на матрицу можно поместить данные полностью и разные объекты не попадают на одни и 
        те же пиксели.\\
    \item Загружены 50 патчей для обучения. Начата обработка данных. \\
\end{enumerate}

Общее количество строк кода: 343\\
\end{document}
